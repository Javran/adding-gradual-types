\renewcommand{\baselinestretch}{1}
\small \normalsize

\begin{center}
\large{{ABSTRACT}}

\vspace{3em}

\end{center}
\hspace{-.15in}


\vspace{3em}

\renewcommand{\baselinestretch}{2}
\large \normalsize


% Gradually gradual typed: program that grows

Type system defines sets of rules, which is checked against programs to
detect certain kind of errors and bugs. Type checking happens either
statically at compile time, or dynamically at runtime. Programming languages
make different choices about type system.

Dynamically typed languages stand out when it comes to scripting and fast prototyping:
programs can be easily modified and executed without need of enforcing consistency and correctness throughout the program.
But as the program grows over time to handle more complicated tasks,
it becomes harder to maintain.

Statically type languages are more reliable and efficient.
Because type errors are caught ahead of execution,
programs are free of these errors at runtime.
Also due to type information are available statically,
more opportunities of optimization can be discovered.
However, a new feature or design change in such program could introduce
great refactoring efforts to not just implementation but also a redesign
of types.

It is only natural that we start to think about a middle ground
between two.
One prevailing approach is gradual typing: in such system,
all type annotations are optional. Programmers can choose to
give types to some values or procedures and later remove them or vice versa.
And the type system will only check portions that are type-annotated,
therefore allows gradual and smoother evolution of programs.

Nonetheless, theoretical works on a gradual typing is far from sufficient
to prove its usefulness: a practical language needs a user base for its community to grow,
efforts are required to develop packages for commonly used data structures, algorithms, etc. Fortunately this problem can be solved by extending existing lanugages,
thus taking advantage of their communities and large amount of libraries.
Several works has done in this regard. And this survey walks through some of them,
discussing challenges and solutions of extending a language to support gradual typing.