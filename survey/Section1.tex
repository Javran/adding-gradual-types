%Section 1
\section{Introduction}

Type system defines sets of rules about programs, which can be checked by a machine
in a systematic way.
One can assign types to appropriate language concepts like
variables, functions and classes to describe expected behaviors
and how they interact with each other.
Then type system takes the responsibility of ensuring that types are respected and
type errors prevent erroneous part of a program from execution.

There are other benefits of type systems besides automated checking:
types can be used to reveal optimization opportunities,
can serve as simple documentation to programmers or
provide hints to development tools to allow auxiliary features like
automate navigation, documentation lookup
or auto-completion.

All these benefits make type system an important part of programming languages,
and there are different ways of accommodating languages with type systems.
The most noticeable difference is the process of checking programs again type rules,
which is also known as type checking.
It occurs either statically ahead of program execution, or dynamically at runtime.
For a static type system, ill-typed programs are rejected by compiler and
no executable can be produced until all type errors are fixed.
For a dynamic type system,
type information is examined at runtime and type errors result
in abortion of the program or exceptions being raised
instead of causing segmentation faults or other worse consequences.
Static and dynamic type system both have their advantages and disadvantages,
and languages make different choices depending on their needs.

In this section, we start off by discussing static and dynamic type systems,
which have inspired many lines of research that combine both within one system.
One instance among them is gradual typing, the main focus of this survey.

\subsection{Strengths and Weaknesses of Static and Dynamic Typing}

The difference between static and dynamic typing is the difference
between whether type checking occurs ahead of execution or at runtime.
This section discusses advantages and disadvantages of both.

\newcommand{\tnum}{\textbf{num}}
\newcommand{\tstr}{\textbf{str}}
\newcommand{\tbool}{\textbf{bool}}
\newcommand{\tarr}[2]{#1 \rightarrow #2}

\subsubsection{Notation}
In the following sections we use $\tau_0, \tau_1, \ldots$ for type variables.
$\tbool$, $\tnum$ and $\tstr$ are types of
boolean values, numbers and strings respectively.\
Tuple types are notated as $(\tau_0, \tau_1, \ldots)$,
and the type notation for functions that takes as input a value of type $\tau_0$
and returns a value of $\tau_1$ is $\tarr{\tau_0}{\tau_1}$.
We use $o : \tau$ to mean that a term $o$ is assigned type $\tau$.
For example $f : \tarr{(\tnum, \tnum)}{\tnum}$ is a function that
takes a tuple of two numbers as input and returns a number.

\subsubsection{Static Type Systems}

A static type system is preferred if robustness and performance
is the main goal of a language.
Because type checking occurs ahead of execution,
ill-typed programs are rejected before it can start, and
when a program typechecks,
one can therefore expect no type error to occur and
no extra cost of typechecking is paid at runtime.

For a static type system to work, all variables, functions, etc. in a program
will need to be assigned types. This is usually done by programmers or through the
means of type inference, which is a technique that infers types using
available type information. This is both an advantage and a disadvantage
of a static type system: having type annotations improves readability
and since programmers are required to keep the consistency between type and code,
type also serves as simple, faithful documentation.
But on the other hand, extra maintenance on type annotations
is required just to allow program execution. This might
be undesirable in situations where a trial-and-error style of programming
is preferable.

Having static known type information also helps in terms of performance.
For example, In machine instructions, primitive arithmetic operations
would only deal with operands of compatible types: addition can only add together
two integers or two floating numbers.
However, in many programming languages,
addition is polymorphic and when it comes to the case of adding an integer
and an floating number, the integer is implicitly converted into a floating number before
performing the actual addition.
For a statically typed system, type information can be used ahead of execution
to figure out exactly whether it is necessary to insert conversions
and where they are required.
But a dynamic typed language might struggle because type information
is only available at runtime.
As a consequence, similar decisions about conversions have to be made at runtime,
imposing performance penalty.

Besides extra effort of maintaining type annotations,
the disadvantage of a statically type language often lies in the lack of runtime flexibility.
If we want to write a program that deals with data whose structure is
unknown at compile time, runtime inspection must be possible.
While typical dynamically typed languages allow inspect of types or object properties,
some extra work are required for a statically typed language
to maintain and check runtime type information.

In addition, it is often less convenient for programmers to prototype
in statically typed languages: it involves trial and error to find the best implementation
to solve a problem, which means the ability to write partial programs,
make frequent changes to structure of variables and functions, but these features are not easily available for a statically typed language.

\subsubsection{Dynamic Type Systems}

Dynamic type systems do typechecking at runtime, which is best suited for
languages designed for scripting and fast prototyping.
A shell script, for example, runs other executables and feeds one's output to another,
effectively gluing programs together.
In such a case, we have little to no knowledge about these executables ahead of time,
making statically type checking improbable.
And when it comes to prototyping,
it is usually more important to allow a trial-and-error type of style of programming
over concerns about robustness and correctness.
In such a scenario, it is more convenient if the type system does not prevent
an incomplete or ill-typed program from execution.

However, it is also easy to spot weaknesses of dynamic type systems.
Despite that dynamically typed languages do not often provide a way of writing type annotations,
programmers usually have facts about behaviors of programs.
These facts are often described in comments or through the naming of variables and functions without a systematic way of verifying them.
As program develops, it is easy to make changes but leaving these descriptions untouched.
This troubles future maintainers to locate the problem and rediscover new facts,
causing maintenance difficulties.

Furthermore, without type information statically available,
dynamic type system needs to maintain runtime information
and rely on it to implement expected semantics.
For example, if it is allowed for a language to condition on non-boolean values,
its type has to be available at runtime in order to
determine how to interpret its value as a boolean at runtime.

Opportunity of optimization could also be obscured for a dynamically type language.
Array cloning, as an example,
can be implemented efficiently as memory copy instructions
knowing the array in question contains only primitive values but no reference or pointers.
But without prior knowledge like this,
every element of the array has to be traversed and even recursively cloned,
which is the usual case for dynamically typed languages without sophisticated mechanism
of optimization.

\subsection{Combining the benefits of the two}

One might have noticed that static type systems and dynamic type systems
are not incompatible in a fundamental level:
the former does type checking ahead of execution while
the latter maintain and inspect type information at runtime.

Indeed, researchers have been looking at different angles of
the possibility of integrating these two type disciplines within one.
Several lines of research are motivated,
and in this section, we will discuss about these research works.

\subsubsection{Design Goals}

Summarizing pros and cons of static and dynamic typed languages,
we now have a picture of a resulting type system integrating the two,
it should be able to:

\begin{itemize}
	\item \textbf{Detect and rule out type errors ahead of program execution.}
	This is one main purpose of having a static type system.
	When a program starts running, type errors should already be eliminated,
	leaving no need of typechecking during execution.
	\item \textbf{Delay typechecking until it is required at runtime.}
	A dynamic type system is known for its permissiveness on what programs it can accept:
	ill-typed or partially written programs are still allowed to execute
	all the way to its termination or until reaching their incomplete or ill-typed parts.
	This makes it suitable in situations where static type information
	is limited or fast prototyping is preferred over robustness or runtime performance.
\end{itemize}

Aside from these two goals,
a statically typed language receives benefits from
utilizing type information for other purposes like
optimization or providing hints to development tools.
For the resulting system, we should be able to have these benefits as well.

\subsubsection{A Introduction to Gradual Typing}

\newcommand{\dyn}{\textbf{dyn}}

Gradual typing \cite{siek2006gradual} is one possible approach that meets our design goals.
It captures the fact that type information might only be partially known
ahead of execution by introducing a special type that indicates partial types.
It attempts to typecheck programs like a static type system does, but delays
typechecking on partial types until sufficient type information is available
at runtime.

In the following sections, we will introduce it
as an extension to static type system.
But it is also possible and practical to support gradual typing by
extending a dynamic type system.
In fact, there are more instances of dynamically typed languages
making extensions to support gradual typing than that of statically typed ones.

\paragraph{Dynamic Type and Type Consistency}

In order to capture the idea that type information might only be partially known
ahead of execution, gradual type system introduces a special type, 
which is called ``dynamic type'' by convention, on top of a static type system
and extends type judgment to allow a more permissive form of type checking in the presence
of dynamic types. For the rest of this survey,
we will use type $\dyn$ to indicate dynamic types
when no language-specific notations are available.

To demonstrate necessary changes to type judgments,
we use statically typed lambda calculus as a starting point.

All type rules are preserved except for function applications,
which are replaced by two rules in gradually typed lambda calculus:

% As for the type judgment extension, all bindings sites are extended to support 

\begin{mathpar}
\inferrule*[left=(GAPP1)]
{\Gamma \vdash_G e_1 : \dyn \\ \Gamma \vdash e_2 : \tau_2 }
{\Gamma \vdash_G e_1 e_2 : \dyn}

\inferrule*[left=(GAPP2)]
{\Gamma \vdash_G e_1 : \tarr{\tau}{\tau'} \\ \Gamma \vdash_G e_2 : \tau_2 \\ \tau_2 \sim \tau }
{\Gamma \vdash_G e_1 e_2 : \tau'}
\end{mathpar}

While GAPP1 allows a value of dynamic type to be typed as if it is a function,
GAPP2 is more interesting: it resembles function application of a static type system,
but instead of demanding $\tau_2 = \tau$, it uses a new relation $\sim$, which
is called type consistency relation.
To deal with dynamic types, type consistency is introduced
to accommodate type judgments for gradual typing.
The following shows rules for type consistency relation.

\begin{mathpar}
\inferrule*[left=(CREFL)]{ }{\tau \sim \tau}

\inferrule*[left=(CFUNC)]
  {\sigma_1 \sim \tau_1 \\ \sigma_2 \sim \tau_2}
  {\tarr{\sigma_1}{\sigma_2} \sim \tarr{\tau_1}{\tau_2} }

\inferrule*[left=(CUNR)]{ }{\tau \sim \dyn}

\inferrule*[left=(CUNL)]{ }{\dyn \sim \tau}
\end{mathpar}

As we can see, if dynamic types are not present at all, type consistency is the same
as type equality. Therefore we still can typecheck all static programs like
static type systems do.
But we now can talk about types with partial information.
Roughly speaking, two types are consistent when there are sensible values that belong to both of them.
We demonstrate this by giving few examples:

\begin{itemize}
	\item $\tbool$ is consistent with type $\dyn$. Despite that the latter is less precise,
	all boolean values can also be of type $\dyn$.
	\item $\tarr{\tnum}{\dyn}$ is consistent with both $\tarr{\dyn}{\tnum}$ and $\tarr{\dyn}{\dyn}$,
	because a function of type $\tarr{\tnum}{\tnum}$ can belong to all these types at the same time.
	\item $\tarr{\dyn}{\dyn}$ is never consistent with $\tbool$, because the former is a function type while the latter a boolean value,
	whose type structures do not match.
\end{itemize}

Now that we have dynamic types, it is no longer a requirement to require
every term in a program to have a type ahead of execution.
For any term missing type information, we can simply assigned it with type $\dyn$,
and type rules will be permissive to accept it as long as type consistency is met.

On the other hand, a dynamically type program can now be statically typechecked
by considering it to be a program without any type annotation.
This is still beneficial, as some literal values and primitives will have their built-in types,
an obvious misuse like taking the square root of a string value can now be detected ahead of execution
instead of raising runtime exceptions.

As a side note, it is intended that type consistency is not a transitive relation: given that $\tau_0 \sim \dyn$ and $\dyn \sim \tau_1$, transitivity will imply $\tau_0 \sim \tau_1$, which would
allow any pair of types to match and therefore render static typechecking useless.

\paragraph{Cast Insertion and Runtime Typechecking}

As we can see gradual type system attempts to contain both statically and dynamically typed programs.
However, because of the presence of dynamic types,
the type system cannot be sound by only have static typechecking.
Imagine function $f : \tarr{\tnum}{\dyn}$ and variable $v : \dyn$, while function application
$f(v)$ can pass static typechecking, the runtime value of $t$ might turn out to be a string
and this violates $f$'s input type $\tnum$.

We can see what have gone wrong: while any number is also of type $\dyn$,
not all values of $\dyn$ type can be numbers.
In general, when we are using a value of $\tau_0$ as if it is a value of $\tau_1$ and
$\tau_0$ is less precise than $\tau_1$
(i.e. when \textbf{downcasting} a value), we need confirmation that this conversion
from $\tau_0$ to $\tau_1$ is safe. This is exactly the idea
behind \textbf{cast insertion} for gradual types.

Casts are functions that when given a value and a type, check whether the value
agrees with the type, then either return the value if it is the case, or report an error
or abort the program.
Before program execution, casts are inserted to guard locations where downcasting happens.
Cast insertion together with static type checking grant soundness for gradual type systems:
a program either terminates successfully, raises type errors ahead of execution or during execution, or
results in error not captured by the type system.

For the previous example, suppose we now have a cast $c : \tarr{\dyn}{\tnum}$.
By inserting the cast, function application $f(v)$ becomes $f(c(v))$.
This still typechecks statically, but whenever we need to pass a runtime value to $f$,
the value is always checked by $c$ to decide whether to pass the value to $f$ or raise
a type error. Now that in the body of $f$,
the type system guarantees that its argument is a number.

\paragraph{Optional Type Annotations and ``pay-as-you-go''}

It is important to realize that gradual type systems are not just about accepting
both fully typed programs and those without any type annotation - it accepts
programs that are partially typed.
This gives programmers more control over what they need from typechecking (\textbf{``pay-as-you-go''}).
As an example, one might wish to implement a function $f$ that expects a string value
as input. There are multiple choices:

\begin{itemize}
	\item We can give $f$ a type annotation:
	$f : \tarr{\textbf{str}}{\dyn}$. By doing so, type system takes the responsibility
	of making sure that the input type is indeed a string.
	This is the only available choice in statically typed languages.
	\item We can do it with our program: right after entering the body of $f$,
	we write code to inspect type of the argument and react accordingly.
	A JavaScript program would do exactly this by branching
	on the result of evaluating \texttt{x \&\& typeof x === "string"} (suppose \texttt{x} is the argument).
	This is often seen in a dynamically typed language, as there could be no other means
	of verification. But if the type system of a gradually typed one does not have
	sufficient type expressiveness to describe desired invariant, this still remains an option.
	\item For time or performance concerns or when programmer expects a frequent change of types
	during prototyping, we might wish to skip all static or runtime checks for a portion
	of a program, which can be achieved by leaving no type annotation. So programmers
	get just what they want from a type system.
	\footnote{A gradual type system might still maintain runtime type information,
	which could impose some overhead. See ``type erasure'' for detail and some viable solutions.}
\end{itemize}

By making typechecking a choice, we have opened up some flexibilities:
for an originally statically typed program, we can now write terms without giving a type,
this allows easy prototyping; for an originally dynamically typed program,
some parts of it might already been stable and robust enough, annotating these parts
with types strengthens it with type error detection ahead of execution, machine-proved correctness,
more opportunities of optimization and 
all other benefits that a statically typed language can enjoy.

\paragraph{Criteria for Gradual Typing}

There are certainly many possible ways of integrating static and dynamic typing
and gradual typing just happens to be one of them.
Siek, Vitousek, Cimini and Boyland's work \cite{siek2015refined}
further introduces \textbf{gradual guarantee} property
to formalize the idea behind gradual typing.
Our intention is not about giving proofs about whether a specific type system is qualified as
gradual typing, but to highlight the goals shared among this particular family of type systems.

In the original work, the following aspects are demonstrated:

\begin{itemize}
	\item \textbf{Gradual typing includes both fully static and fully dynamic}
	A gradually typed language is equivalent to a superset of both a statically typed
	language and a dynamically typed one:
	When a program is fully type-annotated with no presence of dynamic types,
	it should behave the same as its statically typed counterpart;
	on the other hand, a program without any type annotation should behave the same
	as its dynamically typed counterpart. Executing such a program
	results in either a trapped error \cite{luca1997} or successful termination.
	\item \textbf{Gradual typing provides sound interoperability}
	As for partially typed programs, a gradual type system guarantees that,
	if the program typechecks statically, runtime type errors only raises from
	portions with dynamic types.
	This reflects the idea of Wadler and Findler's work\cite{wadler2009well}:
	when a runtime cast fails, the error
	always comes from ``less-typed'' part of the program.
	\item \textbf{Gradual typing enables gradual evolution}
	Adding type annotations moves a program in the direction of static typing
	whereas removing type annotations moves in the direction of dynamic typing.
	Gradual evolution is the idea that programmers can freely add or remove type annotations
	without worrying about changing program behavior, which allows programs to evolve overtime:
	new features can be prototyped with no type annotation and later type can be gradually
	added to the corresponding portion. And old type annotations can be removed,
	giving more room for refactoring or experimenting new ideas without upsetting type system.
\end{itemize}

\subsubsection{Discussion}

Among various research works that combines
benefit of static and dynamic type checking,
gradual typing extensions aim at
making all existing programs written
in the original one still valid with little difference in semantics.
In addition, unlike other approaches, programmers have better control over should and should not be typechecked.

Despite that gradual type systems are well-studied research topics,
extending existing languages to support them is far from trivial task.
To name few differences, while theoretical type systems assume primitive types as simple as just numbers and booleans,
a practical language have various kinds of them including integers, floating numbers, strings and arrays,
which have not been addressed much.
In addition, language-specific features, idioms and common practice needs
to be carefully studied to allow existing code and language users to transit smoothly into
the extended language. In this survey we will explore literatures that extends existing languages to support gradual typing.
In section 2, we will focus on benefits brought by these extensions.
Section 3 discusses and categorizes common and language-specific
challenges researchers have to face and solutions to them.
Related work will be covered in section 4, and
we draw our conclusion in section 5.