\section{Conclusion}

Gradual typing is an approach that brings benefits of static and dynamic type system
together in a sound and efficient manner.
From aspect of programmers, gradual type extensions makes it possible
for dynamically typed languages
to express program invariants in a unified and machine-checkable way,
and for statically typed languages,
programs are allowed to be partially typed to obtain some more flexibility
for situations like refactoring or fast-prototyping.
Additionally, the ability of evolving the program towards
either fully static or fully dynamic programs is also open and intended to be smooth.

In this survey, we motivated gradual typing, visited designs and implementations of Safe TypeScript,
Typed Scheme, Reticulated Python, Gradualtalk and \csharp.
The way towards gradual typing for an existing language is not unique,
because there are different language goals, programming idiom and established user community
to be taken into account.
However, to create a convincing gradual type extension, the way often involves 
not just making type system extensions and choosing dynamic semantics wisely,
but also complementary features that enriches type expressiveness or
reveals more opportunities for optimization or development tool supports.

There are still open questions present in making and maintaining these language extensions and
the majority is about performance and keeping up with original languages.
But this is understandable as many gradual type extensions we have discussed so far
serve as proof of concept or prioritize correctness over performance.
There are considerable amount of room for improvement and we remain confident that
gradual typing extensions are practical and their benefit are significant for existing languages in long run.

\section*{Acknowledgment}

Thanks to David Van Horn for comments, suggestions and discussion regarding
the content of this survey.
Thanks to Jeff Foster, Elaine Shi for their advice and help.
Thanks to my family for their love and support.
Thanks to poi development team and friends from KC3Kai team for
their support and suggestions.

