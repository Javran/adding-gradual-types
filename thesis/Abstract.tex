\renewcommand{\baselinestretch}{1}
\small \normalsize

\begin{center}
\large{{ABSTRACT}}

\vspace{3em}

\end{center}
\hspace{-.15in}


\vspace{3em}
\begin{tabular}{ll}
	Title of dissertation:    & {\large  ADDING GRADUAL TYPES }\\
	&                     {\large  TO EXISTING LANGUAGES:} \\
	&                     {\large  A SURVEY} \\
	\ \\
\end{tabular}

\renewcommand{\baselinestretch}{2}
\large \normalsize


% Gradually gradual typed: program that grows

Type system defines sets of rules, which is checked against programs to
detect certain kind of errors and bugs. Type checking happens either
statically ahead of execution, or dynamically at runtime.
Programming languages make different choices regarding type system to
accommodate their specific needs.

Dynamically typed languages stand out when it comes to scripting and fast prototyping:
programs can be easily modified and executed without need of enforcing consistency and correctness throughout the program.
But as programs grow over time to handle more complicated tasks,
it becomes harder to maintain.
% DUE TO?

Statically typed languages are more reliable and efficient.
Because type errors are caught ahead of execution,
programs are free of these errors at runtime.
In addition, having type information available statically also helps
development in many other ways beyond type system, such as
type-specialized optimization and improved development tools.
However, a new feature or design change in such program could introduce
great refactoring efforts to not just implementation but also a redesign
of types.

It is only natural that we start looking for a middle ground between them.
One of these approaches is gradual typing: in such system,
all type annotations are optional.
Programmers can choose to give types to some values or procedures and later remove them or vice versa, and the type system will be responsible of completely checking only portions
that are type-annotated in both static and dynamic manner.
Gradual typing gives programmer fine grained control over whether type checking should occur,
and guarantees the semantics to be consistent when adding or removing type annotations,
therefore allowing gradual and smoother evolution of programs.

One crucial advantage of gradual typing is that it can be implemented
by extending an existing statically or dynamically typed language:
syntactically, the extension often results in a superset of the original
language that allows more expressiveness in types, which means
the effort of migrating existing code and language users
to take advantage of gradual typing is minimized.
% TODO: and in terms of semantics?

In this survey, we will walk through some research works that
extends existing languages to support gradual types,
discuss challenges and solutions about making these extensions,
and look into related works and the future of gradual typing.