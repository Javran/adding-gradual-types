\begin{abstract}
	There are two kinds of type disciplines of different natures:
	On the one hand, we know that statically typed languages are reliable and efficient,
	but it is not very convenient when it comes to program prototyping or refactoring.
	On the other hand, dynamically typed languages does not prevent programs from execution
	through ahead-of-time checks, which makes them flexible but at the same time
	mistakes become harder to spot and programs more difficult to maintain over time.
	
	It is only natural that we look into
	integrating static and dynamic type systems.
	Among one of the related lines of research is gradual typing, which
	intends to support not just fully static and full dynamic type systems
	but also those partially typed programs.
	By providing control over which part of the program should be checked
	by the type system, programmers are free to evolve programs
	towards either static or dynamic typing in a smooth and consistent manner.

	Despite that gradual type systems are well-studied research topics,
	extending existing languages to support them is far from trivial task:
	to make a successful gradual typing extension,
	the type system is needed to be extended, types to be designed,
	extra features to be introduced and tradeoff to be made, all of which
	require taking into account the design, programming idiom and user community that
	the original language has.
	In this survey, we will walk through the relevant research literature on extending
	existing languages to support gradual types,
	discuss challenges and solutions about making these extensions,
	and look into related works and the future of gradual typing.

\end{abstract}