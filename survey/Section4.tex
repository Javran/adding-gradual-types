\section{Related Work}

Prior to the birth of gradual typing, 

Siek and Taha's work on Gradually Typed Lambda Calculus\cite{siek2006gradual}
establishes the fundamental theory about gradual typing.
By the same authors,
the work is pushed further for object-based languages\cite{siek2007gradual}.
As the concept of gradual typing gains its popularity and being used vaguely to
refer to any work regarding integrating of static and dynamic type system,
Siek, Vitousek, Cimini and Boyland present gradual guarantee\cite{siek2015refined}
to reiterate the intention behind gradual typing and serves a more clear guideline for
languages claimed to be gradually typed.


% \subsubsection{Alternatives to Integrating Static and Dynamic Typing}

There are other type systems that bring benefits of the two together.
Cartwright and Fagan's introduced soft typing, which is a type system
that programers do not write type annotations but use type inference to
assign appropriate types to terms, and runtime checks are inserted as needed.
The drawback of this design is that programmers do not have control over types.
%TODO probably more drawbacks
Quasi-static typing is another attempt, with the idea of dynamic type in use,
it chooses to use a subtyping relation and allows both up-casts and down-casts.
A second pass of plausibility checking detects incompatible casts and signals the program
in question as ill-typed. However its typechecking algorithm did not receive a correct proof,
and it does not statically catch all type errors (TODO: ref)
in the process of attempting the proof, Siek and Taha finds a better solution, 
which results in gradual typing that we have seen today.
%TODO? 

%\subsection{Static type system extensions}


%\subsection{Soft typing}

%\subsection{Like type?}

%\subsection{contracts}

%\subsection{refinement type}

% (TODO) sound gradual typing is nominally alive and well

% TypeScript implements "occurrence typing" (see "Type Guards and Differentiating Types" of advanced types) and Array as tuple 
%\renewcommand{\thechapter}{5}

%\chapter{Future Work}

