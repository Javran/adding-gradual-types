

%\begin{center}
%\large{{ABSTRACT}}

%\vspace{3em}

%\end{center}
%\hspace{-.15in}


%\vspace{3em}
%\begin{tabular}{ll}
%	Title of dissertation:    & {\large  ADDING GRADUAL TYPES }\\
%	&                     {\large  TO EXISTING LANGUAGES:} \\
%	&                     {\large  A SURVEY} \\
%	\ \\
%\end{tabular}

%\renewcommand{\baselinestretch}{2}
%\large \normalsize

\begin{abstract}
% Gradually gradual typed: program that grows

Type system defines sets of rules, which is checked against programs to
prevent certain kind of errors and bugs. Type checking happens either
statically ahead of execution, or dynamically at runtime.
Programming languages make different choices regarding type system to
accommodate their specific needs.

Statically typed languages are reliable and efficient.
Because type errors are caught ahead of execution,
programs are free of these errors at runtime.
In addition, having type information available statically also helps
development in many other ways beyond the type system, such as
type-specialized optimization and improved development tools.
While some believes static type systems assist code evolution in long run,
a new feature or design change in codebase could introduce
great effort to not just implementation
but also adjusting type annotations accordingly, which might not be desirable
for testing out new ideas on top of an existing large codebase.

Dynamically typed languages stand out when it comes to scripting and fast prototyping:
executables from external sources can be used directly without worrying about type information
and programs can be easily modified and executed
without need of enforcing consistency and correctness throughout the program.
But a program grows over time to handle more complicated tasks,
the interaction between different portions of the program becomes harder
to track and maintain.

It is only natural that we start exploring the spectrum between them:
type systems that enjoy benefits of both.
Among one of these is gradual typing, which features a type system
that provides optional type checking.
In such a system, programmer can
give types to only portion of the program.
For the portion that has type information,
static and dynamic typechecking will work together
to ensure type consistency, whereas
the untyped portion of the program is left unchecked,
which might be desirable
when performance or flexibility is concerned.
By giving programmers control over when and where should typechecking
occur, one does not commit to either static or dynamic typing
while losing the benefit of the other.

One crucial advantage of gradual typing is that it can be implemented
by extending an existing statically or dynamically typed language.
Syntactically, the extension often results in a superset of the original
language that allows more expressiveness in types, which means
the effort of migrating existing code and language users
to take advantage of gradual typing is minimized.
Semantically, adding or removing type annotation does not
change the result of programs aside from type errors,
which grants gradual and smooth transition between
static and dynamic type disciplines.

In this survey, we will walk through the relevant research literature on extending
existing languages to support gradual types,
discuss challenges and solutions about making these extensions,
and look into related works and the future of gradual typing.

\end{abstract}