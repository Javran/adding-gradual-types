%Section 2

\section{Benefits of Gradual Typing}

% TODO (???)
Gradual typing not just allows language users to write partially typed programs,
it also make it possible to bring in features meant for static or dynamic typing.
In fact, these extras are no less important than gradual typing itself,
as it helps make the language more convenient to use
and therefore motivates programmers to use gradual typing to their advantages.

In this section, we will explore some literatures
from aspects of language users and see in detail about these benefits
gradual typing have granted us.

\subsection{From JavaScript to Safe TypeScript}

Starting as a scripting language,
JavaScript has grown into a language that powers many large web applications.
Unfortunately, as a language that exists over a long period of time,
the language evolves but many of its flaws are still left as it is for compatibility reasons,
hindering productivity of even most experienced programmers.

Among tools and extensions to JavaScript that attempt to ease this pain,
TypeScript is one closely related to gradual typing.
Its syntax is a superset of JavaScript that supports optional type annotations\cite{bierman2014understanding}.
A compiler typechecks TypeScript source code
and compile it into plain JavaScript,
making it ready to be executed out of box for JavaScript interpreters.

However, TypeScript is intentionally unsound:
it is designed to typecheck programs only statically
and removes all traces of type when emitting code in JavaScript.
Therefore invariants about types cannot be enforced at runtime
 and benefits from the type system is limited.

Nonetheless, TypeScript still serves as a reasonable starting point
for Safe TypeScript by Rastogi, Swamy, Fournet, Bierman \& Vekris\cite{rastogi2015safe}.
Safe TypeScript shares the same syntax with TypeScript,
but features a sound gradual type system and efficient implementation\footnote{It is suggested
that Safe TypeScript only cover a large fragment of JavaScript, but we assume that this coverage
is sufficient to support all essential features of JavaScript.
}.

In this part we will visit language features of Safe TypeScript and
see how it improves JavaScript.

\subsubsection{Type Checking for Free}

Despite JavaScript is capable of examining type of values at runtime,
it does not have support for type annotations.
This leads to a common practice in which some lines of code are placed
right after entering the body of a function
to check its argument types, report errors if any before
proceeding with actual implementation.
This can be seen in the following code snippet:

\begin{minted}{javascript}
function f(x) {
    if (typeof x !== 'string') {
        // throw error
    }
    return x + '!';
}
\end{minted}

As a toy example, the function accepts a string value,
then returns another with exclamation sign concatenated to it.
This does not look complicated, but imagine in real projects,
there would be multiple arguments to a function and some of them have to
go through this process of checking.
Without machine-checkable and consistent way of enforcing these invariants,
it is easy to make mistakes while maintaining code written in such a style.

By extending the language with type annotations,
we can do something better in Safe TypeScript:

\begin{minted}{javascript}
function f(x : string) : string {
    // now this check becomes unnecessary
    if (typeof x !== 'string') {
        // throw error
    }
    return x + '!';
}
\end{minted}

In the code above, we just declared a function $\texttt{f}$ that accepts
one variable $x$ of $\texttt{string}$ type, and returns a value of $\texttt{string}$ type.
Safe TypeScript then performs typechecking and insert casts when needed to ensure that,
when $\texttt{f}$ is called, its argument is indeed a $\texttt{string}$.
This renders the $\texttt{typeof}$ check at the beginning unnecessary
and therefore can be safely removed:

\begin{minted}{javascript}
function f(x : string) : string {
    return x + '!';
}
\end{minted}

The use site of $\texttt{f}$ might look like the following:

\begin{minted}{javascript}
f('one'); // typechecks
f(1); // does not typecheck
function g(x) {
    return f(x); // might be unsafe
}
\end{minted}

Here Safe TypeScript is able to tell that:
calling $\textbf{f}$ with a string literal
is safe, and no extra cast is needed; the second call is immediately rejected
because number literal is clearly not a $\texttt{string}$;
and for the third case where $\texttt{x}$
is not explicitly given a type, Safe TypeScript compiler will insert runtime type cast
and either allow it to proceed to the body $\texttt{f}$ when $\texttt{x}$ is indeed a string,
or throw an error before even entering the body of $\texttt{f}$.

Notice that by making use of type annotations, Safe TypeScript not only allows
clearer code, but also able to spot some type errors ahead of execution:
imagine the function application $\texttt{f(1)}$, when its written in plain JavaScript,
we have to wait for the check in function body to throw an error,
while annotating $\texttt{x}$ with a type allows Safe TypeScript
to spot the error even before execution.

This benefit of detecting type errors earlier is not a specific one for Safe TypeScript.
In general, adding a phase of static typechecking to a dynamically typed language
might render some runtime checks no longer necessary.
Once these checks are removed,
programs might end up running more efficiently without compromising correctness.


\subsubsection{Object-Oriented Programming in Safe TypeScript}

Although JavaScript does not support classes\footnote{
	Classes were not standard features of JavaScript
	as of publication date of Safe TypeScript.
},
its prototype-based model allows essential features of object-oriented
programming to be simulated in JavaScript.
Designed to be a superset of JavaScript, Safe TypeScript must accommodate such a programming
style as well.

TypeScript introduces the concept of interface to allow describing
the structure of runtime objects and Safe TypeScript follows this concept as well.
The following example defines
interface \texttt{Point}, which is a valid type for objects
that has two fields \texttt{x} and \texttt{y} whose types are all \texttt{number}s:

\begin{minted}{javascript}
interface Point { x: number; y: number }

function f(o: Point) {
   // omitted
}
f({x: 1, y: 0}); // typechecks
f({x: 1, y: "a"}); // does not typecheck
f({x: 1, y: 0, z: 2}); // typechecks
f(1); // does not typecheck
\end{minted}

As demonstrated by the example above,
interface can only typecheck with objects that has all expected fields of expected types.

Classes are given special treatment\footnote{While TypeScript treats classes structurally,
	this does not match with JavaScript semantics. See ``Nominal Type, Structural Type and Subtyping''
	of section 3.3 for detail.
} to cooperate better with model used by JavaScript. To be more precise,
it treats class types nominally but allows it to be viewed structurally.

In addition to the \texttt{Point} interface defined above,
suppose we have the following definition:

\begin{minted}{javascript}
class CPoint implements Point {
  constructor(public x: number, public y: number) {
    this.x = x; this.y = y;
  }
}

function f1(v : {x: number, y: number}) { return true; }
// f2 is equivalent to f1
function f2(v : Point) { return true; }
function f3(v : CPoint) { return true; }

var obj1 = {x: 1, y: 2}; var obj2 = new CPoint(1,2);
f1(obj1); // typechecks
f1(obj2); // typechecks 
f3(obj1); // does not typecheck
f3(obj2); // typechecks  
\end{minted}

This example demonstrates few features of interest:

\begin{itemize}
	\item \textbf{Structural view of classes}
	Since interfaces are structural, \texttt{f1} and \texttt{f2} are equivalent.
	In addition,
	objects created from object literals and instance created from classes can both typecheck against it.
	In the example above, despite that \texttt{obj1} and \texttt{obj2} are created in different ways,
	they both have expected fields of expected types and therefore function calls \texttt{f1(obj1)}
	and \texttt{f1(obj2)} do not raise any type error.
	\item \textbf{Interface declaration}
	In fact, \texttt{implements Point} is optional since interfaces are structural.
	It is nonetheless a good practice to write it out.
	Because this not only serves as a simple documentation,
	but it also allows Safe TypeScript to understand programmer's intention
	so type system can warn user when a declared interface is not actually satisfied.
	\item \textbf{Nominal view of classes} When using class names as types,
	Safe TypeScript treats them nominally, this is consistent with
	JavaScript's object model and \texttt{instanceof} operator, as the latter does not allow a structural
	match.
	Therefore while \texttt{f3(obj2)} typechecks, \texttt{f3(obj1)} does not.
\end{itemize}

\subsection{From Scheme to Typed Scheme}

Scheme is another dynamically typed language used
for casual scripting as well as industrial applications.
Instead of relying on any specific type discipline, programmers
reason about their code in informal ways,
which makes the language highly flexible.
But for the same reason,
it becomes a real challenge to design gradual type systems
that can work well with scheme programs.

Tobin-Hochstadt and Felleisen's work on Typed Scheme\cite{tobin2008design}
provides a novel and practical solution.
In the same paper, the notion of occurrence typing is introduced,
which cooperates with Scheme well to
form the building block of a gradual type system for Scheme.

\subsubsection{Support Informal Reasoning}

The following code\footnote{Most of the examples in this section are adapted from \cite{tobin2008design}.}
shows a function definition as typical style of programming in this language:

\begin{minted}{scheme}
;; a Complex is either
;; - a number
;; - (cons number number)
(define (creal x)
  (cond [(number? x) x]
        [else (car x)]))
\end{minted}

It defines a function \texttt{creal}, which takes as argument
a complex number.
Rather than relying on type systems, the contract is informally written in the comment,
which indicates that input value is supposed to be either a number or
a pair of numbers representing real and imaginary part respectively for a complex number.
Predicate\footnote{By Scheme convention,
	a predicate is a procedure that always return a boolean value.
} \texttt{number?} distinguishes number representation from the other one.
So in first clause of \texttt{cond}, we know for sure that \texttt{x} is a number,
whereas the second clause deals with remaining case,
so \texttt{x} must be a pair with proper use of \texttt{car}
to extract its real part.
There are few key observations from this example:

\begin{itemize}
	\item Difference occurrences of the same variable might have different set of possible values.
	\item This style of informal reasoning relies on result of applying predicates, and 
	a predicate distinguishes one particular set of values from others.
	For example, \texttt{number?} can distinguish numbers from other values.
	\item By reasoning about the execution path taken to reach certain point in the program,
	predicates about a variable can be accumulated to further constraint the set of possible values.
\end{itemize}

%TODO: such

These observations suggest that we could use predicates as some sort of types
when extending the language with gradual typing: say all values that make predicate \texttt{number?}
return true are \texttt{Number}s, and all values that make predicate \texttt{cons?} return true
are pairs, we should be able to encode this complex number representation using types.
In fact, Typed Scheme version of it begins with such a type definition:

%It is important for the type system to be able to assign same variable with different
%types depending on the context where the variable appears, this is exactly the notion of
%occurrence typing.
%Since these informal reasonings rely on predicates used in control flows,
%the idea is to use predicates as building block for types.

%The Typed Scheme version begins with definition of complex number:

\begin{minted}{scheme}
;; a Complex is either
;; - a number
;; - (cons number number)
(define-type-alias Cplx (U Number (cons Number Number)))
\end{minted}

Note that \texttt{U} constructs a union type, it combines multiple types together so
a value satisfying any of these types will also satisfy the resulting union type.
As a separate note,\texttt{Cplx} is just an alias for the union type,
it nonetheless improves readability
and allows other parts of the program to refer to it by name.

The body of the function in Typed Scheme looks like:

\begin{minted}{scheme}
(define: (creal [x : Cplx]) : Number
  (cond [(number? x) x]
        [else (car x)]))
\end{minted}

% occurrence typing ref?

Note that despite \texttt{x} is of type \texttt{Cplx} as it enters the function body,
it is actually a number in first clause of \texttt{cond} and a pair of numbers
in the second clause. This phenomenon is captured by the idea of
\textbf{occurrence typing}: one variable can have different types
for its different occurrences, and the type depends on execution path it takes to reach that location.
The use of predicate \texttt{number?} allows type system to know that \texttt{x}
in the first clause must be a number. By eliminating number type from \texttt{Cplx},
the type system can further conclude that in the second clause \texttt{x} must be a pair of numbers.

% TODO: ref LTI
Thanks to local type inference supported by Typed Scheme,
explicitly giving the return type \texttt{Number}
is sufficient to let the type system know that every clause of \texttt{cond} returns a number
so this part of the code can typecheck successfully.

% TODO: XXX

Another different kind of reasoning is also being used among Scheme programmers:
Suppose we have stored a list of various types of values in \textbf{xs}
and the following code will compute the sum of all numbers from \textbf{xs}:

\begin{minted}{scheme}
(foldl + 0 (filter number? xs))
\end{minted}

Note that despite \textbf{xs} is a list that contains not just numbers,
it is guaranteed that \textbf{(filter number? xs)} will return a list of numbers
so \textbf{foldl} will work properly to produce the desired result.

The expression requires no modification at all to typecheck in Typed Scheme
and it receives type system benefits: suppose \textbf{map} is mistakenly
used instead of \textbf{filter} or \textbf{number?} is replaced by
other predicates insufficient to test whether a value in question is indeed a value,
the type system is able to recognize such errors and give warnings.

\subsubsection{Refinement Type}

Note that in Scheme, we use \textbf{number?} to distinguish numbers
from other type of values.
In general, given a boolean-valued unary function
we can define the set of values that produces \textbf{true} when fed to this function.
This allows Typed Scheme to support refinement type.

\begin{minted}{scheme}
(: just-even (Number -> (Refinement even?)))
(define (just-even n)
  (if (even? n) n (error 'not-even)))
\end{minted}

The code above uses boolean-valued function \textbf{even?}
and its corresponding refinement type \textbf{Refinement even?},
so we can expect a value of this type to be not just numbers, but also only even ones.

More practical examples include use refinement types to distinguish raw input
from validated ones:

\begin{minted}{scheme}
(: sql-safe? (String -> Boolean))
(define (sql-safe? s) ...)
(declare-refinement sql-safe?)
\end{minted}

In this example, user defines function \textbf{sql-safe?} to verify
that a raw string contains no SQL injection or other contents of malicious attempt.
Then a refinement type is declared using this very function.
This makes available type \textbf{(Refinement sql-safe?)} to rest part of the program
to avoid mistakenly using raw input rather than verified safe ones.

As we can see occurrence typing not just enables
gradual typing, but also allows refinement type,
which provides us a powerful tool to use user-defined functions
to define custom types of better precision.

\subsection{From Python 3 to Reticulated Python}

Python 3 (Python for short) is another popular dynamic typed language.
M. Vitousek, M. Kent and Baker's work on Reticulated Python (Reticulated for short)
explored extending Python with gradual typing.

Python comes with annotation syntax.
This syntax allows expressions to be optionally attached to
function definitions and their arguments,
which makes it suitable for type annotation.

Given proper definition of \textbf{Int}, we can write annotated
function \textbf{distance} like below:

\begin{minted}{python}
def distance(x: Int, y: Int)-> Int:
  return abs(x - y)
\end{minted}

While annotation allows arbitrary expressions, Reticulated Python
only uses types built from several type constructors.

\subsubsection{Function Types}

Python 3 has different ways of calling the same function.
Suppose we want to use \textbf{distance} defined above,
we can use positional arguments like \textbf{distance(3,4)},
or use keywords like \textbf{distance(y=4, x=3)} and the result should be the same.

To support keyword calls, \textbf{Named} constructor is used to give
\textbf{distance} type \textbf{Function(Named(x: Int, y: Int), Int)}.
To support traditional positional arguments, \textbf{Pos} is used instead.
The type system also allows \textbf{Named} to be used as if it is constructed by \textbf{Pos}
when their length and element types correspond. So functions can be called in both
ways without problem. Additionally \textbf{Arb} is another constructor
that allows functions of arbitrary parameters.

\subsubsection{Class and Object Types}

Python is an object-oriented programming language.
Programmers define classes and create objects from them.
Both classes and objects are runtime values in Python
and Reticulated provides corresponding types for both.

Consider the following example:

\begin{minted}{python}
@fields({'x': Int})
class 1DPoint:
  def __init__(self: 1DPoint):
    self.x = 0
  def move(self: 1DPoint, x: Int)->1DPoint:
    self.x += x
    return self

p = 1DPoint()
\end{minted}

It defines a class \textbf{1DPoint} with one field \textbf{x} of \textbf{Int} type.
And an object is created from it through the constructor
and is bound to variable \textbf{p}.

Reticulated derives class type from this definition:

\begin{verbatim}
Class(1DPoint){
  move: Function(Named(self: 1DPoint, x: Int), 1DPoint)
}
\end{verbatim}

And object types are similar:

\begin{verbatim}
Object(1DPoint){
  move: Function(Named(x: Int), 1DPoint)
}
\end{verbatim}

While object can access methods of its class, it is also possible
to assign methods and fields to objects to change its behavior.
Therefore Reticulated makes object type open with respect to consistency.

\subsubsection{Dynamic Semantics}

Reticulated Python is not only a practical gradual type extension,
but serves as an experiment of exploring different dynamic semantics

\subsubsection{Other Features}

There are other features of Reticulate.
In particular, since it is not always possible to know what module would be loaded
at runtime, static typechecking sometimes occurs right after module is loaded.

Despite Python syntax supports annotation at function definition,
it does not provide a way of annotating local variables with tyoe.
Instead, Reticulated takes a step further and perform dataflow-based type inference.


\subsection{From Smalltalk to Gradualtalk}

Smalltalk is a language of highly dynamic nature:
it supports live programming, which means programs with incomplete
methods should be accepted, and programmers in Smalltalk relies on
idioms that are tricky to type properly.
Allende, Callau, Fabry, Tanter and Denker's work on Gradualtalk shows
us a well combination of various features in order to extend Smalltalk
into a gradual typed language.

\subsubsection{Annotating Programs with Types}

In Smalltalk there are objects that receives and processes messages.
The following example computes euclidean distance for points:

\begin{minted}{smalltalk}
Point >> distanceTo: p
| dx dy |
dx := self x - p x.
dy := self y - p y.
^ (dx squared + dy squared) sqrt
\end{minted}

Gradualtalk extends this syntax that allows
type annotation on parameter, return value and local variables.

\begin{verbatim}
Point >> (Number) distanceTo: (Point) p
| dx dy |
dx := self x - p x
dy := self y - p y
^ (dx squared + dy squared) sqrt
\end{verbatim}

Since local variable \textbf{dx} and \textbf{dy} does not explicit type-annotated,
they receive \textbf{Dyn} as default type.

Blocks are basic features in Smalltalk,
whose corresponding types are available in Gradualtalk in the form of normal function types.

\begin{verbatim}
Polygon >> (Number) perimeter: (Point Point -> Number) metricBlock
...
\end{verbatim}

This definition expects \textbf{metricBlock} to be block
that takes as argument two points, and returns a number.

\subsubsection{Self Types}

One practice in Smalltalk is to return object itself for setters
to allow chained calls:

\begin{verbatim}
Point >> (Point) y: (Number) aNumber
y := aNumber.
\end{verbatim}

Notice that if we want to extend this class in future,
its setter with be tagged with type \textbf{Point} and some
type information will be lost in the way.
To deal with this problem, Gradualtalk introduces self type:

\begin{verbatim}
Point >> (Self) y: (Number) aNumber
y := aNumber.
\end{verbatim}

If any other class inherits from it, the self type will make sure
to point to that class instead of sticking with \textbf{Point},
this allows type information to be preserved.

\subsubsection{Union Types}

Gradualtalk supports union types, which is useful when
source code contains different branches and each of them might return different types.

\begin{minted}{smalltalk}
Boolean >> (a | b)
  ifTrue: (-> a) trueBlock
  ifFalse: (-> b) falseBlock
\end{minted}

Note that \textbf{trueBlock} and \textbf{falseBlock} might return
different types, allowing this method to have \texttt{a} or \texttt{b}
will not be satisfactory, and simply using \texttt{Dyn} type
returns in a lose of information.
Gradualtalk solves this problem by introducing union types: \texttt{a | b}
is a type by itself, which is compatible with both \texttt{a} and \texttt{b}.

\subsubsection{Structural and Nominal Types}

A structural type describes the shape of an object.

\begin{minted}{smalltalk}
RBParser >> bracketsOfNode: ({left (-> Integer) . right (->Integer)}) node
\end{minted}

The method above defines the argument type of \texttt{node} to have 2 methods:
\texttt{left} and \texttt{right} and both of them will return an integer when invoked.

A nominal type, on the other hand, is induced by classes
(for example an instance of \texttt{String} will have \texttt{String} type).
In addition, subtyping relations of them are formed from existing inheritance relation.

Gradual type managed to unify them in an interesting way.

\begin{minted}{smalltalk}
RBParser >> bracketsOfNode: (RBNode {
left (-> Integer) .
right (->Integer)
}) node
\end{minted}

Note that in addition to the structural type we have seen,
\texttt{RBNode} is a nominal type, the type of \texttt{node} above
limits the value to be not just an instance that understands certain methods,
but also requires it to be an instance of \texttt{RBNode}.

Another interesting example is the combination of structural type
and nominal type \texttt{Dyn}:

\begin{minted}{smalltalk}
Canvas >> (Self) drawPoint: (Dyn {x (-> Integer) . y (->Integer)}) point
... point x. "safe call"
... point y. "safe call"
... point z. "not an error, considering point to be Dyn"
\end{minted}

Besides calling \texttt{x} and \texttt{y}, a call to \texttt{z} does not
raise an error ahead of execution, because \texttt{Dyn} can be casted to allow use of
\texttt{z}.

\subsubsection{Type Aliases}

Gradualtalk supports type aliases. This does not expand expressiveness of type itself,
but it make reusing existing types convenient.

With type aliases comes the concepts of protocols, which are just type aliases
for structural and nominal types.

\newcommand{\csharp}{C^\sharp}
\subsection{From $\csharp 3.0$ to $\csharp$ 4.0}

While we have seen many examples that a dynamically typed language gets
gradual type extensions, Bierman, Meijer and Torgersen's work on
C$^\sharp$ 4.0 shows us a practical example of statically typed language
extends towards gradual typing.
This extension makes interoperation between statically and dynamically typed code
more convenient and brings in features meant for dynamically typed languages.

\subsubsection{Interaction with Dynamic Objects}

The following code snippet shows how C$^\sharp$ 3.0 interact with JavaScript:

\begin{minted}{csharp}
...
ScriptObject map = win.CreateInstance("VEMap", "myMap");
map.Invoke("LoadMap");
...
string latitude, longtitude;
...
var x = win.CreateInstance("VELatLong", latitude, longitude)
var pin = map.Invoke("AddPushpin", x);
pin.Invoke("SetTitle", name);
...
\end{minted}

In order to interact with JavaScript, method calls
are made using string-based interface, which is verbose, fragile to maintain.

With C$^\sharp$ 4.0, the source code can be simplified to:

\begin{verbatim}
...
dynamic map = win.CreateInstance("VEMap", "myMap");
map.LoadMap();
...
string latitude, longtitude;
...
var x = win.CreateInstance("VELatLong", latitude, longitude)
var pin = map.AddPushpin(x);
pin.SetTitle(name);
...
\end{verbatim}

Notice how types are changed to \textbf{dynamic} and methods are called
as if there are regular objects.
These improvements are more than just syntactic: gradual typing is implemented
so that typechecking can be involved and this part of code can benefit from it.

\subsubsection{Using Features Intended For Dynamic Languages}

The extension also makes available some features meant for dynamic languages
directly in C$^\sharp$, one example is the use of \textbf{ExpandoObject}

\begin{verbatim}
...
dynamic contact = new ExpandoObject();
contact.Name = "Erik"
contact.Address = new ExpandoObject();
contact.Address.State = "WA";
...
\end{verbatim}

For a statically typed language, to maintain structured object,
one either needs to declare a record type with matching structure,
or use a dictionary in which one key and one value type should be declared.
However, with the introduction of \textbf{ExpandoObject},
by simply assigning values to properties of it,
they are brought into existence without boilerplate about type casts.
This helps in terms of fast prototyping, in which programmers cares more about
having a working implementation than maintain type precision.

