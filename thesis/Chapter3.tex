%Chapter 3

\renewcommand{\thechapter}{3}

\chapter{Challenges and Solutions}

Having explored syntactic improvements to gradual type extensions,
this chapter discusses how there are implemented to unveil common and language-specific
challenges and solutions about them.

\section{Extensions to Type Systems}

Supporting Gradual type system consists of two parts.
First, it requires traditional static types including basic type, function type and types for structural data.
Second, it is necessary to introduce a dynamic type to allow incomplete type information to present in programs.

\subsection{Extending From Static Type Systems}

\subsection{Extending From Dynamic Type Systems}

\section{Dealing with Structural Data}

Structural data combines data types together to form organized data for processing.
It is such an important concept that all languages we have discussed so far study
it to some extent. The notion can also be extended to object-oriented programming,
in which structural data are just objects with only fields and objects are also
allowed to have methods who can either share methods from other objects,
or defines it own to allow different runtime behavior.

\subsection{Structural Data and Subtyping}

\subsection{Object-Oriented Programming}

\subsection{Nominal or Structural}

\subsection{Recursive Data Types}

using predicates
constraints as type level computation (requires a type system of sufficient expressiveness)

typing structural data

nominal
- structural

object identity

giving `this` / `self` special type treatment

inheritance (could merge with subtyping)

variable / member mutation

\section{Other Performance Concerns}

cast insertion

Large array

runtime check overhead

\section{Language-specific Challenges}

%Chapter 4
\renewcommand{\thechapter}{4}
\chapter{Related Work}

(TODO) sound gradual typing is nominally alive and well


TypeScript implements "occurrence typing" (see "Type Guards and Differentiating Types" of advanced types) and Array as tuple 
\renewcommand{\thechapter}{5}

\chapter{Future Work}

\renewcommand{\thechapter}{6}
\chapter{Conclusion}