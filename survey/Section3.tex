%Section 3

\section{The Way Towards Gradual Typing}

In previous chapter we have explored the benefits gradual typing
can bring in for existing languages.
Now we takes a step further to discuss the way towards making these extensions.

Despite that we have gradual guarantee\cite{siek2015refined} and other
theoretical works to serve as guidelines for gradual typing extensions,
there is no canonical way of supporting gradual typing
and design choices have to be made
depending on various facts like original type discipline, common practice, performance and even user community.

Additionally, adding gradual types to an existing language
is not just an effort of putting theories of gradual typing into practical use,
but also introductions of new language features
that take the advantage of gradual typing
and help development or improve performance in these languages.

In this chapter, we will explore various issues regarding
type system extensions, design choices,
implementation, and other technical details.
This allows us to obtain more understanding and insight about the way towards gradual typing for existing languages.

\subsection{Extensions to Type Systems}

For statically typed languages, gradual type system introduces
dynamic types, which then allows presence of partial type information
in types.
For dynamically typed languages, gradual type extension
enables type annotations or enriches its type expressiveness
to allow programmers to enforce invariants in a more uniform and
machine-checkable way.
While type soundness is established by a cooperative work
of static and dynamic typechecking,
gradual typing itself opens up possibility of some other interesting extensions
that makes the extended language more convenient to use and more attractive to programmers.

These extensions can be split into two categories:
\textbf{essential extensions} are those necessary
for a complete gradual typing support;
and \textbf{complementary extensions} are those that improves type expressiveness
but a type system without them does not compromise completeness.

\subsubsection{Essential Extensions}

We are looking at research works that falls into type disciplines of two different nature:
JavaScript, Scheme, Python 3 and Smalltalk are all dynamically typed languages, from which
much similarity shows up.
On the other hand, we have \csharp\ 3.0, a statically
typed langauge.

\paragraph{Extension from Dynamically Typed Languages}

The approach used by Safe TypeScript and Reticulated Python
is to first perform static typechecking in the presence of dynamic types.
If it succeeds, programs written in extended languages are then translated into original languages.
The translated program could contain runtime checks with some values instrumented,
these checks and instruments are either implemented by inserting expressions or statements in place or
making calls to external libraries\footnote{
	These libraries are also known as ``runtime''s
	and often shipped
	with the gradual type extension to support
	runtime checks.
}.

This approach grants high compatibility with original languages: source code written in extended language
can be compiled and then used as if it is written in the original language, and thanks to the fact
that all existing dynamically typed languages
we are discussing have decent supports for modern package managers,
including runtime support is as simple as making a dependency declaration
for the packages using the extension.
Besides, by compiling to the original languages,
we can be confident to some extent that the language implementations
are independent of any specific implementation of the original languages,
albeit some language extensions do make assumption about implementations of their original languages\footnote{
	Part of Python is written in C, which does not respect the indirection that
	Reticulated proxies use and a work around requires removing and re-installing
	proxies at the boundary of Python and C\cite{vitousek2014design}.
}.

Gradualtalk is an extension to Pharo Smalltalk. It consists of 3 parts:
the core allows representing types in Smalltalk, the typechecker
is responsible of performing static typechecks and finally
a type dictionary for storing type information.

Typed Scheme takes a similar approach: the implementation is a macro
that takes source code in Typed Scheme, does static typechecking and
inserts casts when needed and translates the result.
This phase either results in failure in static typechecking, in
which case the detected type error needs to be addressed,
or successful translated code in PLT Scheme with casts inserted
in the form of contracts.

\paragraph{Extension from A Static Type System}

The work on \csharp\ 4.0 shows us how statically typed language can be extended
to support gradual typing.
This approach is novel because \csharp\ 3.0 uses a bidirectional type system \cite{pierce2000local}
rather than a standard static type system.
There are two distinct phase in a bidirectional type system:
in \textbf{typechecking} phases, we determine whether a particular type can be
assigned to a given term, whereas in \textbf{type synthesis} phases,
we are given a term and tasked to determine a type for it.
As bidirectional type checking is a variant of coercive subtyping\cite{breazu1991inheritance},
both phases generate translated terms with explicit coercions.
During the process of checking types for a \csharp\ program,
these two phase naturally interleaves each other to progressively
assign types to terms and turn implicit coercions into explicit ones.
Upon successful completion of this process, all coercions are made
explicit and dynamic types are no longer needed and simply replaced by basic object types.

As shown in Siek's work in \cite{siek2007gradual},
making dynamic type the top type in subtyping order allows any two types to
be coercible. \csharp\ 4.0 avoids this problem by
not allowing dynamic types to be implicitly converted to any other types
during typechecking phases.
It turns out that this approach is sufficient to achieve
transitivity of subtyping
even when dynamic types are present.

The type system of \csharp\ 4.0 shows that it is possible for a
static type system to allow transitivity of subtyping when making gradual type extensions.
But whether this technique is applicable to standard static type system is still remained
as an open question.

\subsubsection{Complementary Extensions}

While essential extensions establish the completeness of gradual type systems,
complementary extensions are not exactly requirements.
After all, programmers can always choose not to use explicitly typed terms and
leave types of them inferred or defaulted to dynamic types.
This is however unsatisfactory, as using dynamic types often result in a lost of type information in the process.
Therefore extra features of type systems are usually introduced to improve type expressiveness and
in general make language more convenient to use.

\paragraph{Union Type}

Practical languages utilize control flows to be able to determine
the sequence of operations to perform base on runtime values.
This allows same block of code to return values of different types and simply
assigned returned value to a most general type results in information loss.

Imagine the following code in Typed Scheme:

\begin{minted}{scheme}
(lambda (x: number)
  (if (> x 0) x #f))
\end{minted}

Depending on the test \texttt{(> x 0)}, we either get a number or the boolean value \texttt{\#t} as result,
which can be expressed through union type \texttt{($\cup$ Number Boolean)}.

Union type is a simple extension employed by Typed Scheme and Gradualtalk
to solve this problem.
By notation, an union type $U = \tau_0 \cup \tau_1 \cup \ldots$ is a superset of
all $\tau_x$s involved. For any value $v$, $v : U$ typechecks as long as $v : \tau_x$ does.
By performing flow analysis on a block of code, we can find a list of types of return values to
then construct a union type to allow preserving more type information.

\paragraph{Type for Objects}

Object allows related values and codes to be organized together and is an important concept
in various languages. Many language rely heavily on object-oriented programming,
which demands proper type system support for them.
Siek and Taha pioneered gradual typing for objects \cite{siek2007gradual}, which
treats objects structurally as a pack of methods with labels.
However, when applying to existing languages,
this approach results in various issues that goes beyond type system.
We will have a separated section regarding these issues.

\paragraph{Occurrence Typing}

Supporting gradual typing requires having matching types for values and functions in the languages.
Most extensions we have discussed have concepts of classes and objects, they are typed either
structurally or nominally.
However, instead of using classes or objects like other languages do,
typical Scheme programs prefers using composed data structures as simple as just pairs, vectors
and they are distinguished through testing the shape of the value or symbol comparison.
To solve this potential mismatch between values and types, occurrence typing is introduced.
The idea is that the difference occurrences of the same variable would have different types
depending on the context where it occurs.
Types are defined by specifying some boolean-returning unary functions as type discriminators.
When such a function \texttt{p?} in question returns non-falsy value for a value \texttt{v},
the type system will consider \texttt{v} to have value \texttt{p}.
This way, a \texttt{if}-expression that depends of value of \texttt{(p? v)},
will have its two branches with different types of \texttt{v}, one with type \texttt{p}
and another with non-\texttt{p} type.

Despite this extension is shown only in Typed Scheme, we believe it can be applied to other languages
as well: the latest version of TypeScript
has implemented \textbf{type guard}\cite{ts2018},
which uses this exact technique to improves
the expressiveness of its type.

\paragraph{Parametric Polymorphism}

The idea behind parametric polymorphism
(a.k.a. generic programming in some languages)
is that some functions and structures can
work uniformly regardless of the value it is operating on.
Besides its presence in \csharp\ 3.0,
Safe TypeScript, Typed Scheme, Reticulated Python and Gradualtalk all include
this feature to further extend their type expressiveness.

The following illustrates this feature in Safe TypeScript:

\begin{minted}{scheme}
interface Pair<A,B> { fst: A; snd: B }
function pair<A,B>(a: A, b: B}: Pair<A,B> {
  return {fst: a, snd: b};
}
\end{minted}

In this example, \texttt{Pair} is a structure that contains two pieces of data and
\texttt{pair} a function as its constructor.
The two pieces of data can be retrieved
by accessing \texttt{fst} and \texttt{snd} attributes without specific knowledge
of actual data types. Type variables are \texttt{A} and \texttt{B}, which
allows getting data types back once \texttt{Pair<A,B>} is known.

This feature is made possible in Safe TypeScript by extending typing context with type variables
and allowing type abstraction at appropriate places
(e.g. in interfaces or function declarations as shown above).
In Gradualtalk, same feature is implemented based on
Ina and Igarashi's work\cite{ina2011gradual}.

\subsection{Cast Insertion}

Cast insertion provides the mechanism for runtime typechecking.
Casts are functions that takes a single value, checks whether the value is of expected type
then either throws type error or proceed the program with the value just checked.
Static typechecking for a gradual type system has an extra purpose of inserting casts:
when dynamic types are encountered during static typechecking, the type information
at hand is insufficient to guarantee type safety. Therefore casts are inserted
in appropriate places so they can be checked at runtime.

Taking Safe TypeScript as an example:

\begin{minted}{typescript}
function toOrigin(q: {x: number, y: number}) {q.x = 0; q.y = 0;}
function toOrigin3d(p: {x: number, y: number, z: number}){
  toOrigin(p); p.z = 0;
}
toOrigin3d({x: 17, y: 0, z: 42})
\end{minted}

The code above is translated into JavaScript:

\begin{minted}{javascript}
function toOrigin(q) { q.x = 0 ; q.y = 0; }
function toOrigin3d(p) {
  toOrigin(RT.shallowTag(p, {"z": RT.num})); p.z = 0;
}
toOrigin3d({x: 17, y: 0, z: 42})
\end{minted}

in which \texttt{RT} is Safe TypeScript runtime.
Notice how argument of \texttt{toOrigin} requires an object that contains
both \texttt{x} and \texttt{y} attributes while in the body of
\texttt{toOrigin3d}, it only tags object \texttt{p},
which is known to have \texttt{x}, \texttt{y} and  \texttt{z}, with extra attribute \texttt{z}.
Thanks to static typechecking, we have reduce the amount of runtime checks
by skipping checking types already known statically.

While static typechecking is performed during compilation,
dynamic typechecking is embedded in code and therefore impose a runtime overhead over original programs.
To make the performance of partially typed programs close with its dynamically typed counterparts,
It is essential to reduce the amount of dynamic checks.
For value of primitive types, we can do no better than simple cast insertions,
but there are still space when it comes to structured data and functions,
researchers have come up with different strategies to balance between type safety, efficiency or design simplicity. We will explore thees strategies in next section.

\subsection{Blame Tracking}

When a cast involves function types, it cannot be checked immediately\cite{findler2002contracts,siek2009exploring}.
For example, it requires function applications to provide an argument list to check argument types
of a function type, and checks on return types cannot be done before the function in question
returns. Since these checks are delayed, it is possible that when a runtime cast fails,
information necessary to local the actual problem is far way from it.

Gradualtalk uses blame tracking\cite{wadler2009well} to solve this problem:
when a check is delayed, the type system keeps track of the information,
which allows blames to be assigned properly when the check fails.
In Reticulated, guarded dynamic semantics implements blame tracking:
the static type system provides error messages and line numbers,
which is kept in the proxies so that actual source of error can be identified properly.

\subsection{Object-Oriented Programming}

Object-oriented programming is a popular style of programming.
All languages we have discussed so far supports it to some extent.
Among these languages,
some even have the language itself designed to embrace its philosophy.
To have a sound and efficient gradual typing support for it
becomes center of topic.

\subsubsection{Unified Concepts}

Despite that each language have a slightly different model of objects, they do have concepts
in common.
In this section, we give definition to some important concepts in object-oriented programmming
to allow discussing them in an unified way.

An \textbf{object} $o$ is a single value that contains \textbf{fields} (somethings also called
\textbf{properties} or \textbf{attributes}) and \textbf{methods}.
Unique labels $l$s are assigned to all fields and methods. By using notation $o.l$, we can \textbf{access}
(i.e. \textbf{read} from or \textbf{write} to) fields and methods of object $o$.
Fields are values, and methods are functions within whose body gain access to a special variable
(called \texttt{this} or \texttt{self} by convention) which can be used to refer to the object itself.

A special kind of singleton objects are called \textbf{classes}.
For a class $C$, we can create (a.k.a. \textbf{instantiate}) objects by calling constructor.
The created object is considered an \textbf{instance} of class $C$.

Classes can form a partial order of subclass relations: if a class $C_0$ \textbf{extends} from
another class $C_1$, we say $C_0$ is a \textbf{subclass} of $C_1$ and $C_1$ a \textbf{superclass} of $C_0$.
Methods can be shared through the mechanism of class-instance and inheritance relation:
when we try to access $o.l$, $l$ is looked up in the order of instance itself (some languages disallow
objects to have their own methods therefore skipping this step), its class, and superclass of its class,
all the way through this chain of inheritance until hitting the top object.
The lookup resolves to the first successful one,
but if no method matching $l$ is found, program throws an error to inform about an unknown method being invoked.

Sometimes the concept of \textbf{interfaces} (or \textbf{protocols}) is used. It is a set of
field or method labels with associating any values or functions to them.
Any object that can resolve all labels list by an interface is considered to
\textbf{implement} that interface.

\subsubsection{Nominal Type, Structural Type and Subtyping}

There are two different approaches of assigning objects with types:

\begin{itemize}
	\item \textbf{Nominal types} are derived from class definitions, these types usually have a matching name with its
	corresponding class.  Nominal types follow the traditional sense of inheritance:
	one must explicitly extend from another class to make it a subclass of the other
	(with a commonly assumed exception that topmost class is always a superclass of other classes).
	\item \textbf{Structural types}, on the other hand, views an object structurally.
	Inheritance, or rather subtyping, in this case is closer to an interface: if one object can resolve all labels of another object, it is considered a subtype of it. This approach is sometimes known as duck typing.
\end{itemize}

Despite that these two approaches coincide with each other in many ways,
they are not actually compatible.
And due to the highly dynamic nature of some languages,
it will not be sufficient to simply having either structural and
nominal types.
Therefore researchers often end up with a hybrid approach
to allow coexistences of these two approaches.

To illustrate the problem, suppose we have the following definitions in Python
(assuming all methods will return values of the same type):

\begin{minted}{python}
class A:
  def foo(self):
    ...

class B(A):
  def bar(self):

class C:
  def foo(self):
    ...
  def bar(self):
    ...
\end{minted}

While a nominal type allows a variable $v : A$ to be assigned an instance of $B$,
assigning an instance of $C$ to $v$ is forbidden by the type system: despite
having all methods that $A$ requires, it is not explicitly extending $A$ to form a proper relation of
inheritance.

Structural type is more of a relaxed relation in this case: $A$ is simply $\{foo: \tau \}$ (for whatever appropriate type $\tau$), $B$ and $C$ are both equivalent to $\{foo: \tau, bar: \tau\}$. Therefore structurally, $B$ and $C$ are both subtypes of $A$ despite no being explicitly declared.

Another interesting example comes from TypeScript.
Suppose we have the following definition\footnote{This example is taken and adapted from \cite{rastogi2015safe}.}:

\begin{minted}{javascript}
interface Point { x: number; y: number }

class MovablePoint implements Point {
  constructor(public x: number, public y: number) {
    this.x = x; this.y = y;
  }
  public move(dx: number, dy: number} {
    this.x += dx; this.y += dy;
  }
}

function mustBeTrue(x : MovablePoint) {
  return !x || x instanceof MovablePoint;
}

// typechecks, but returns false
mustBeTrue({
  x: 1, y: 2,
  move: function(dx,dy) {
    ...
  }
});
\end{minted}

One might expect calls to function \texttt{mustBeTrue} to always return \texttt{true},
but in TypeScript this is not the case.
As TypeScript treats classes structurally, any object that has field \texttt{x}, \texttt{y} and \texttt{move}
of matching types can typecheck with \texttt{MovablePoint} without being an actual instance of it.
So \texttt{instanceof} operator, which only respects the prototype chain,
is inconsistent with TypeScript's type system.

To solve this problem, Safe TypeScript treats class types nominally but allows it to be viewed structurally.
Taking the same example but assume that we are using Safe TypeScript,
instances of \texttt{MovablePoint} can still be used in places where type
\texttt{\{x: number, y: number, move(dx: number, dy:number): any\}}
or \texttt{\{x: number, y: number\}} is expected.
But in order to typecheck a value against \texttt{MovablePoint},
the value now has to be an instance of \texttt{MovablePoint}.

Among other languages we studied,
Typed Scheme extends its structure system and uses nominal types \footnote{Thanks to the expressiveness of refinement type, it is still possible to test an object structurally.};
Reticulated Python chooses the structural approach - despite class inheritance being allowed,
duck typing is generally considered idiomatic approach in Python.
On the other hand, Gradualtalk takes an approach similar to that of Safe TypeScript's:
it implements a unified syntax to allow an object to be typed nominally and structurally at the same time.

\subsubsection{Objects and Dynamic Semantics}

While most of extensions we have studied so far agrees on how casts are inserted
around primitive values as originally suggested in \cite{siek2006gradual},
objects and dynamic semantics are always center of discussion among languages
that relies heavily on it.

For this part, our discussion follows the work on Reticulated Python,
as Reticulated is not just an implementation of gradually typed Python,
but also a testbed of 3 different dynamic semantics, which covers most of the cases.
Along the way, approaches from other literature of similar nature will be discuss and added as well.

There are 3 different approaches implemented by Reticulated, known as \textbf{guarded}, \textbf{transient}
and \textbf{monotonic}.

For starter, an example in Reticulated is given:

\begin{minted}{python}
class Foo:
  bar = 42
def g(x):
  x.bar = 'hello'
def f(x:Object({bar: Int})->Int:
  g(x)
  return x.bar
f(Foo())
\end{minted}

Note that function \texttt{g} mutates the type of \texttt{x.bar} therefore its callee \texttt{f}
no longer have a \texttt{x} of proper type to return \texttt{Int}.
When different dynamic semantics are applied, the result varies as well and in following parts
we will visit these approaches in order.

\paragraph{The Guarded Dynamic Semantics}

This approach wraps actual objects in a proxy which builds itself up using sequences of casts.
This keeps proxies to always be one step alway from the actual object rather than building
a chain of proxies that compromises efficiency over time.
Method invocations and field accesses are relayed to the actual object,
field writes and return values of method invocation are checked at the boundaries
of statically and dynamically typed code.

After translation, the relevant functions have casts inserted as following:

\begin{minted}{python}
def g(x):
  cast(x, Dyn, Object({bar: Dyn})).bar = 'hello'
def f(x:Object({bar: Int}))->Int:
  g(cast(x, Object({bar: Int}), Dyn))
  return x.bar
f(cast(Foo(), Object({bar:Dyn}), Object({bar:Int})))
\end{minted}

When the control reach body of function \texttt{g},
\texttt{g} will have the most precise type \texttt{Object({bar:Int})} inferred from the sequence
of operations. And the write to field \texttt{bar} will try casting a string into \texttt{Int},
which results in a failure.

Note that this approach have the problem that object identity is not preserved:

\begin{minted}{python}
x is cast(x, Dyn, Object({bar:Int}))
\end{minted}

As \texttt{cast} function wrap \texttt{x} in a proxy, it is no longer considered the same object as \texttt{x}
therefore returns \texttt{False}. However instance tests will still work, as proxies are considered a subclass.
And for a similar reason, type test on such a value might fail because from outside it is a \texttt{Proxy}
rather than an instance of the relevant class.

\paragraph{The Transient Dynamic Semantics}

Instead of using proxies, the transient approach inserts casts at use sites or at sites where
the value becomes relevant. Under this approach, the same Reticulated function is translated as following:

\begin{minted}{python}
def g(x):
  cast(x, Dyn, Object({bar: Dyn})).bar = 'hello'
def f(x:Object({bar: Int}))->Int:
  check(x, Object({bar: Int}))
  g(cast(x, Object({bar: Int}), Dyn))
  return check(x.bar, Int)
f(cast(Foo(), Object({bar:Dyn}), Object({bar:Int})))
\end{minted}

There are two differences: first, despite having the same name, \texttt{cast} is a different
function that just checks whether cast on object is allowed and then returns it instead
of wrapping it in a proxy; second, calls to \texttt{check} are inserted around the body of \texttt{f}.
In this dynamic semantics, it is the final call to \texttt{check} inside the body of \texttt{f} that
detects type error and throws, as \texttt{x.bar} is mutated and its type no longer matches the return
type of \texttt{f}.

\paragraph{The Monotonic Dynamic Semantics}

The monotonic approach relies on a slightly strong assumption than the other two approaches:
if any update happens to object fields, it will not change its type.
Instead of using a proxy, the object keep type information of its own. As casts are execute on objects
overtime, its type information gets locked down to always become more precise.

The translation is the same as that of guarded dynamic semantics, but with \texttt{cast} being a different
function that applies monotonic approach.

Right before call to \texttt{f}, object newly created from \texttt{Foo()} is locked down to
have a type at least as precise as \texttt{Object({bar:Int})}, this locks down type of \texttt{x.bar}
therefore raises an error when the body of \texttt{g} tries to write a mismatching type to it.

A similar approach is used by Safe TypeScript, in which runtime type information (RTTI) is maintained on
objects themselves. And as program proceeds, RTTI decreases with respect to the subtyping relation,
which avoids some unnecessary runtime checks to be performed over and over again.
Additionally, another improvement done by Safe TypeScript is
the strategy of shallowly tagging objects in RTTI, this avoids tagging objects that are never touched
at the cost of having to propagate RTTI when needed, in practice researcher finds it to be a good tradeoff.

\paragraph{Type Dictionary}

This approach is used by Gradualtalk, but can also be adapted to other languages of similar nature:
in Smalltalk, certain classes are not allowed to be modified because
the virtual machine relies on them to implement fundamental behaviors.
Therefore language implementor maintains a type dictionary that keeps type information for objects
instead of tagging directly on object themselves.
This can be considered to be strategy for prototyping gradual typing extension for a language when
efficiency is not a urgent concern as work on Gradualtalk is not (yet) focusing on performance either,
and the tagging strategy is not being stated in detail to give more insight.

\subsubsection{RTTI and Differential Subtyping}

Runtime Type Information (RTTI) is one part of the mechanism for Safe TypeScript
to implement its dynamic typechecking. Much like the monotonic approach found in Reticulated Python,
with casts inserted in appropriate places and objects instrumented, runtime casts can be performed.

One key contribution of Safe TypeScript is its use of differential subtyping:
it is an extension that not only tells subtyping relations, but also allows splitting
a structural type into slices of interest.
The observation is that, during static typechecking, when subtyping relation is used, there is a potential loss in precision. We can extend subtyping relation to compute the exact set of type information,
and encode in the form of RTTI instead of keeping types of all fields around therefore improve performance.

\subsubsection{Reference \texttt{this} or \texttt{self}}

In object-oriented programming, the body of class methods
have access to a special reference,
which is usually named \texttt{this}\footnote{Some languages would use \texttt{self} as its name.}.
This reference allows the method to refer to the instance of the class and its members.

As we have seen in Section 2, simply assigning the class type in which the method is defined
to \texttt{this} reference
does not always capture the most precise use of \texttt{this}.
For example,
if one class $C_1$ is extended from class $C_0$, calling methods
defined in $C_0$ will only allow \texttt{this} to be assigned type $C_0$, ignoring
the fact that the class is extended and $C_1$ should be a more precise type for \texttt{this}.

This problem is recognized and solved in the work of
Reticulated Python, Gradualtalk and Safe TypeScript, 
In Reticulated, \texttt{self} is bound separately from all other method arguments\footnote{In Python,
method types are supposed to use \texttt{self} as the first argument explicitly.}
to allow it to refer to the most precise class at hand;
Gradualtalk adapts ``self type'' from the work of Saito and Igarashi\cite{saito2009self}
and use type system to allow it to be resolved to the current class.
On the other hand, Safe TypeScript is a prototype-based language and have a different
semantics for \texttt{this} reference.
Its type system recognized the fact that \texttt{this} is passed implicitly
in function calls and keeps track of it.
In addition, \texttt{this} is also given special treatment
so that when invoking a method of \texttt{this}, type system is capable of
searching the prototype chain to find it.

\subsection{Type Inference}

Programs with type inference and those with gradual typing have some similarity in the sense
that both allow programmers to write less type annotation themselves.
Many languages we have mentioned so far supports type inference to some extent.
TypeScript implements local type inference within method bodies,
as Safe TypeScript is built on top of it, it inherits this features as well.
Reticulated Python implements local type inference by
using dataflow analysis to compute possible types for each variables
and then join these types to determine what type should be assigned to each variables.
A local type inference is also employed in Typed Scheme.

\subsection{Type Erasure and Optional Typing}

Sometimes it is helpful to ensure that no runtime overhead is imposed
on some values: for a tag-based implementation of gradual typing,
an large array of objects will suffer from unnecessary taggings.
In such cases, some languages provide a way to ensure that no tag is attached
to these objects.
Note that this is different from using dynamic types:
when using dynamic types, as we have seen in dynamic semantics,
some implementation of gradual typing allows structure of types to become more precise
during execution, an erased type ensures that no runtime tagging will happen
therefore typing is avoided.
Of course, without tagging it is impossible to enforce some invariants at runtime,
which remains as a choice that programmers can do.

\subsection{Challenges}

There are still challenges and open questions present when
making gradual type extensions. This section attempts to
organize them into common challenges and language-specific ones.

\subsubsection{Common Challenges}

\paragraph{Keeping up with the original language}
One problem of making extensions to an existing language
lie in the fact that the extended one needs extra maintenance efforts to keep being
a superset of the original language.
If the language being extended is actively changing or extending its syntax,
our gradual typed version will have trouble keeping it up or
even run into the problem of conflicting syntax.
This is particular true in case of JavaScript: nowadays ECMAScript Next
proposals are still being reviewed and some of them will eventually
be accepted into the language,
which requires Safe TypeScript to bring in these new features
and careful adjustment might be required so that the syntax for
optional type annotation do not accidental conflict with the original language.

\paragraph{Libraries maintenance}
Despite that gradual typed systems claim to provide freedom of making choices
between statically or dynamically typed,
for this feature to be useful, type signatures must be provided
and kept up-to-date, which might require extra maintenance effort.
As an example, DefinitelyTyped\cite{yankov2014definitelytyped}
is a repository that contains TypeScript type definitions for popular
JavaScript libraries and it has to be actively maintained to
match its JavaScript counterpart.

\paragraph{\texttt{eval} function}

Languages of highly dynamic nature would provide a \texttt{eval} function,
which when given a string or a abstract syntax tree, evaluates it as
if it is one part of the program and returns the result.
It is difficult to assign types\footnote{To be more precise,
we cannot determine any information about \texttt{eval} function's return type.
} to these functions, as
the resulting type depends on the input and running
a complete analysis on input is undecidable.
Despite that \texttt{eval} function is provided
for almost all languages of dynamic nature, the problem is minor.
This is because the use of such a function is generally considered
temporary or bad practice, development tools will give warnings
regarding any use of \texttt{eval} and ask programmers to try alternative approaches.

\paragraph{Problems with translation}
It is a popular approach that an extended language have programs
written in it translated into its original language.
In some sense this is a fragile approach as the runtime library
could be modified to break its runtime checking mechanism,
and if the original language provides interface for inspecting
objects, information intended to be hidden could be
revealed and modified. This brings negative consequence for a tag-based
gradual type extension.

Sometimes interoperability has an impact on translation.
CPython implementation of Python relies on calling C functions to
achieve better performance, which breaks proxies used by Reticulated.
The workaround is to remove and re-install proxies
between language boundaries, which is not satisfactory.

In addition,
despite that we can establish type soundness in the extended language,
a translated program will still be unnecessarily checked by the original language.
For instance,
while Safe TypeScript maintains its own static typechecking phase,
it is implement as a plugin to TypeScript compiler,
which is incapable of turning off TypeScript's own static typechecking.
A similar situation happens to Reticulated, as it cannot turn off
runtime checks impose by Python itself.

\subsubsection{JavaScript}

Originally designed for casual scripting and extended over a long period of time,
JavaScript admittedly contains many design flaws that
programmers must be aware of and work around them.
Any language extension claimed to fully support JavaScript inherits these problems as well.
Some goes to the path of writing another language that compiles to JavaScript,
therefore completely avoiding dealing with JavaScript directly for programmers.
Other like Safe TypeScript, only support a reasonable subset of JavaScript -
some use of language features are considered deprecated or bad practice,
so not fully supporting all programs that JavaScript accepts does not raise
much concern in practice.

\subsubsection{Scheme}

Typical Scheme practice involve heavy use of macros.
Despite Typed Scheme is able to handle most of the macros by expanding them before typechecking,
some macro systems of sufficient complication require some understanding of their own invariants,
which remains future work.

Another challenge that scheme faces is that the type expressiveness of Typed Scheme
not sufficient for some highly flexible functions:
some functions can be of indefinite arity, for example \texttt{(map f xs)}, \texttt{(map f xs ys)}
and \texttt{(map f xs ys zs)} can all be valid given \texttt{f} of expected arity;
a similar situation is with \texttt{apply} function, which accepts a function and a list of arguments.
Typed Scheme has special cases to deal with these cases, but it is more desirable
if these types can be expressed using the type language.

\subsubsection{Smalltalk}

Smalltalk features a live system: developers might modify code, run it and debug it using the same execution environment, this raises several challenges in the development of Gradualtalk:

\begin{itemize}
	\item \textbf{Granularity of the compiling process}
	Smalltalk's compile unit is smaller than that of many other languages.
	Instead of doing per-class compilation, Smalltalk compiles in the unit of method to
	allow a live system. This causes troubles when defining methods with circular dependencies.
	Gradualtalk solves this problem by decoupling typechecking and compiling process,
	so a type error does not prevent compilation of a incomplete program.
	\item \textbf{Work done by the typechecker can be obsolete}
	When new methods are added or type signature for an existing type changes,
	typecheckers can produce obsolete results. This is solved by using a dependency tracking
	mechanism and with the use of ghost entities.
	\item \textbf{Type errors in critical code} Sometimes type errors can be critical:
	for example, if default error handling function itself contains some kind of error,
	this would result in an infinite loop. This is solved by allowing cast insertion
	to be disabled.
\end{itemize}
